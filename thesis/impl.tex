\section{Sage implementations of formulas and test cases}
Below we provide Sage implementations and test suites
for some of the calculations this paper.

\subsection{Orbital and Gross-Keating implementations and tests}
\subsubsection{Description of implemented functions}
\subsubsubsection{Semi-Lie orbital integral in $S_2(F) \times V'_2(F)$}
All the formulas in \Cref{sec:semi_lie_intro_formulas} are implemented.
\begin{itemize}
  \ii \texttt{O(r, vb, vc, ve, vda)}
  implements \Cref{thm:semi_lie_formula}.

  \ii \texttt{delO(r, vb, vc, ve, vda)}
  implements \Cref{cor:semi_lie_derivative_single},
  divided by a factor of $\log q$.

  \ii \texttt{delO\_combo(r, vb, vc, ve, vda)}
  implements \Cref{cor:semi_lie_combo},
  divided by a factor of $\log q$.
\end{itemize}

\subsubsubsection{Inhomogeneous orbital integral for $S_3(F)$}
The next part of the program implements the orbital integral
that is alluded to in \Cref{thm:summary}.
The parameters are those described in \Cref{lem:S3_abcd} and \Cref{lem:parameter_constraints}.
\begin{itemize}
  \ii The $\Arch$ function defined in \Cref{def:arch} is implemented
  as \texttt{ARCH(a0, a1, w1, w2, k)}.
  \ii There are some common sums appearing in the formulas
  \Cref{thm:full_orbital_ell_odd,thm:full_orbital_ell_even,thm:full_orbital_ell_neg}
  which are implemented as follows.
  \begin{itemize}
    \ii The function \texttt{ARCH\_sum\_n(a0, a1, w1, w2)} implements the sum
    \[ \sum_{k=a_0}^{a_1} \left( 1 + q + \dots
      + q^{\Arch_{[a_0, a_1]}(w_1, w_2)(k)} \right) (-q^s)^k \in \ZZ[q^s, q]. \]
    \ii The function \texttt{ARCH\_sum\_c(a0, a1, w1, w2)} implements the sum
    \[ \sum_{k=a_0}^{a_1} \Arch_{[a_0, a_1]}(w_1, w_2)(k) (-q^s)^k \in \ZZ[q^s]. \]
  \end{itemize}
  \ii Using these functions, we implement \texttt{O\_for\_S3(r, l, delta, lam)}
  as the full orbital integral for all the cases.
  Hence this function implements all there of
  \Cref{thm:full_orbital_ell_odd,thm:full_orbital_ell_even,thm:full_orbital_ell_neg}.
\end{itemize}

\subsubsubsection{Derivative of the inhomogeneous orbital integral for $S_3(F)$}
We also implement functions that can compute the
derivative of the orbital integral for $S_3(F)$.

\begin{itemize}
  \ii \texttt{ARCH\_deriv\_n(r, C, W, H)} implements \Cref{lem:derivative_nn}.
  \ii \texttt{ARCH\_deriv\_c(r, C, W, H)} implements \Cref{lem:derivative_cc}.
  \ii \texttt{delO\_for\_S3\_via\_arch(r, l, delta, lam)} implements
  $\frac{1}{\log q}\partial\Orb(\gamma, \phi)$
  by calling the previous two functions,
  using the arguments that are mentioned in the proof of \Cref{thm:group_kernel_full}.
  \ii \texttt{delO\_for\_S3(r, l, delta, lam)} implements the final formula for
  $\frac{1}{\log q}\partial\Orb(\gamma, \phi)$
  detailed in \Cref{thm:S3_orbital_deriv}.
  It is the result of directly substituting \Cref{lem:derivative_nn} and \Cref{lem:derivative_cc}
  with the parameters described in the proof of \Cref{thm:S3_orbital_deriv}
\end{itemize}

\subsubsubsection{The geometric side}
\begin{itemize}
  \ii \texttt{gross\_keating\_sum(n1, n2)} implements the sum in \Cref{prop:GK}.
  \ii \texttt{GK(r, vb, vc, ve, vda)} implements the sum $\GK(r, v(b), v(c), v(e), v(d-a))$
  that we introduced in \eqref{eq:GKdef}.
  \ii \texttt{clean\_intersection(r, vb, vc, ve, vda)} implements the
  right-hand side of \Cref{thm:clean_intersection},
  although we use the argument names corresponding to the matched elements
  after the translation in \Cref{lem:finale_match}.
\end{itemize}

\subsubsection{Randomized test suite}
The code then implements the following tests to check correctness.
These tests are randomized tests where the parameters are randomly selected by a program
which can then numerically compute them;
they do not purport to be symbolic or formal proofs of the formulas.

\begin{enumerate}
  \ii \texttt{test\_O} verifies that the formula \Cref{thm:semi_lie_formula}
  matches the casework described in \Cref{ch:orbitalFJ1}.
  To speed up the test, rather than symbolically comparing,
  it selects the values $q = 17$ and $s = \log_q 1337$ in comparing the sides.

  Within this test, some auxiliary functions are defined.
  \begin{itemize}
    \ii The function \texttt{O\_brute\_odd(r, vb, vc, ve, vda)}
    is a na\"{\i}ve implementation of the casework in \Cref{ch:orbitalFJ1} when $\theta$ is odd.
    \ii The function \texttt{O\_brute\_even(r, vb, vc, ve, vda)}
    is a na\"{\i}ve implementation of the casework in \Cref{ch:orbitalFJ1} when $\theta$ is even
    (encompassing both \textbf{Case 5}, \textbf{Case 6\ts+} and \textbf{Case 6\ts-}).
  \end{itemize}

  \ii \texttt{test\_delO} verifies that \Cref{cor:semi_lie_derivative_single}
  matches the derivative of \Cref{thm:semi_lie_formula}.
  \ii \texttt{test\_delO\_combo} verifies that \Cref{cor:semi_lie_combo}
  matches a subtraction of \Cref{cor:semi_lie_derivative_single}.

  \ii \texttt{test\_matrix\_upper\_triangular} verifies that the important entries
  of the matrix in \Cref{lem:semi_lie_ker_full_rank} are computed correctly
  (that is, one indeed gets an upper triangular matrix
  when looking at the relevant rows).
  \ii \texttt{test\_kernel\_large\_r} verifies \Cref{lem:semi_lie_large_r} holds.
  \ii \texttt{test\_kernel\_full} verifies \Cref{thm:semi_lie_finite_codim_full} holds.

  \ii \texttt{test\_O\_for\_S3} verifies that the formulas
  \Cref{thm:full_orbital_ell_odd,thm:full_orbital_ell_even,thm:full_orbital_ell_neg}
  match the casework described in \Cref{ch:orbital1}.
  To speed up the test, rather than symbolically comparing,
  it selects the values $q = 17$ and $s = \log_q 1337$ in comparing the sides.

  There are many subfunctions in here used for the
  na\"{\i}ve implementations of the casework.
    \begin{itemize}
      \ii \texttt{O\_zero(r, l, delta)} implements $I_{\le 0}$ as in \Cref{prop:O_zero}.
      \ii The function \texttt{vol\_1disk(n, vxx, rho)} implements \cref{lem:volume}.
      Here the argument \texttt{vxx} corresponds to $v(1 - \xi \bar \xi)$.
      \ii The function \texttt{vol\_2disk(n, vxx, rho)} implements \cref{lem:quadruple_ineq}.
      Here the argument \texttt{vxx1} corresponds to $v(1 - \xi_1 \bar \xi_1)$
      and the argument \texttt{vxx2} corresponds to $v(1 - \xi_2 \bar \xi_2)$.
      \ii The function \texttt{qs\_weight(n, m)} implements the weight
      \[ \varkappa \cdot (-1)^n q^{s(2m-n)} q^{2n-2m} \Big( q^{2m}(1-q^{-2}) \Big) \]
      that occurs frequently throughout.
      This expression is the product of the factor
      $\varkappa \cdot (-1)^n q^{s(2m-n)} q^{2n-2m}$ from \Cref{sec:qs_weight_from_sub}
      and the volume factor $\Vol(t_1: -v(t_1)=m) = q^{2m} (1-q^{-2})$.

      \ii The function \texttt{O\_case\_1\_2\_brute(r, l, delta, lam=None)}
      is a na\"{\i}ve implementation of \textbf{Case 1} and \textbf{Case 2}
      from \Cref{ch:orbital1}.
      For odd $\ell$, we combine it with $I_{\le 0}$ in \texttt{O\_ell\_odd\_brute(r, l, delta)}
      to get a na\"{\i}ve implemenattion of \Cref{thm:full_orbital_ell_odd}.
      For even $\ell$ we instead get (together with $I_{\le 0}$) is \Cref{prop:ell_even_half}.

      \ii The function \texttt{O\_case\_3\_4\_brute(r, l, delta, lam)}
      is a na\"{\i}ve implementation of
      \textbf{Case 3\ts+}, \textbf{Case 3\ts-}, \textbf{Case 4\ts+}
      and \textbf{Case 4\ts-} (asserting it never occurs)
      from \Cref{ch:orbital1}.
      Putting all the cases together gives
      \texttt{O\_ell\_odd\_brute(r, l, delta, lam)}
      which is a na\"{\i}ve implementation of \Cref{thm:full_orbital_ell_even}.

      \ii The function \texttt{O\_ell\_neg\_brute(r, l, delta, lam)}
      is a separate na\"{\i}ve implementation of \Cref{thm:full_orbital_ell_neg}.
      It re-does the cases separately in the same was as this paper
      rather than using the previous brute-force implementation.
    \end{itemize}

  \ii \texttt{test\_delO\_for\_S3\_via\_arch} verifies that the derivatives of the formulas in
  \Cref{thm:full_orbital_ell_odd,thm:full_orbital_ell_even,thm:full_orbital_ell_neg}
  match those predicted by \Cref{lem:derivative_nn} and \Cref{lem:derivative_nn}.
  Hence it can be thought of as a verification of those two lemmas.

  \ii \texttt{test\_delO\_for\_S3} verifies \Cref{thm:S3_orbital_deriv}.
  \ii \texttt{test\_ker\_for\_S3} verifies \Cref{thm:group_kernel_full}.
  \ii \texttt{test\_GK\_to\_orbital} verifies \Cref{thm:miracle}.
  \ii \texttt{test\_clean\_intersection} verifies \Cref{thm:clean_intersection}.
\end{enumerate}

\subsubsection{Code listing}

\singlespacing
\lstinputlisting[language=Python]{checkthesis.sage}
\doublespacing

\subsection{Quaternion implementations and tests}
As an afterthought we also provide the following short self-contained file
verifying the quaternion calculations done in \Cref{ch:jiao}.
Unlike the previous code, it is symbolic.

\singlespacing
\lstinputlisting[language=Python]{checkquat.sage}
\doublespacing
